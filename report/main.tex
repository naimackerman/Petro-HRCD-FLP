% =================================================================
% INCOSE Conference LaTeX Template V1.2 (Release Date: November 4th, 2025)
% Copyright (c) 2025 INCOSE
% 
% This template is provided for use in preparing manuscripts for
% INCOSE conferences. You may use, modify, and 
% distribute this template for academic and professional purposes.
% 
% This template is provided "as is" without warranty of any kind.
% The author(s) disclaim all warranties, express or implied,
% including but not limited to warranties of merchantability and
% fitness for a particular purpose.

% =================================================================

\documentclass[11pt,letterpaper]{article} % Remove this line if using A4 size.
%\documentclass[11pt,a4]{article} % A4 is also accepted and is typically used for non-US submissions.

% ---------- Core layout ----------
\usepackage[
  letterpaper,
  left=0.6in,right=0.6in,top=0.6in,bottom=0.6in,
  headheight=85pt,headsep=-45pt
]{geometry}
\raggedbottom
\usepackage{graphicx}
\graphicspath{{./}{figures/}}
\usepackage{float}
\usepackage{amsmath}
\usepackage{multicol}
\usepackage{tikz}
\usetikzlibrary{positioning, arrows.meta}
\usepackage{booktabs}
% \usepackage{natbib}
% \usepackage[letterpaper,top=2cm,bottom=2cm,left=3cm,right=3cm,marginparwidth=1.75cm]{geometry}
% Useful packages
\usepackage{amssymb}
% \usepackage[colorlinks=true, allcolors=blue]{hyperref}


% ---------- Captions & float spacing ----------
\usepackage{caption}
\captionsetup{
  font={sf,bf,footnotesize},
  labelfont={sf,bf,footnotesize},
  justification=centering,
  labelsep=period,
  hypcap=false
}
\setlength{\textfloatsep}{8pt}
\setlength{\floatsep}{6pt}
\setlength{\intextsep}{8pt}
\setlength{\abovecaptionskip}{12pt}
\setlength{\belowcaptionskip}{1pt}

% ---------- Tables / lists ----------
\usepackage{booktabs}
\usepackage{array}
\usepackage{tabularx}
\usepackage{enumitem}
\setlist{nosep}
\usepackage[table]{xcolor}
\definecolor{tableheader}{HTML}{D9D9D9}
\newcolumntype{P}[1]{>{\sffamily\centering\arraybackslash}p{#1}}
\newcolumntype{Y}{>{\sffamily\centering\arraybackslash}X}
\newcolumntype{A}[1]{>{\raggedright\arraybackslash}p{#1}}
\newcolumntype{M}[1]{>{\raggedright\arraybackslash}m{#1}}

% ---------- Fonts ----------
\usepackage[T1]{fontenc}
\usepackage[utf8]{inputenc}
\usepackage{PTSerif}
\usepackage[scaled]{helvet}
\renewcommand{\sfdefault}{phv}
\newcommand{\headingfont}{\sffamily}

% ---------- Headings ----------
\usepackage{titlesec}
\setcounter{secnumdepth}{0}

% Heading 1
\titleformat{\section}
  {\headingfont\bfseries\raggedright\fontsize{18pt}{18pt}\selectfont}{}{0.75em}{}
% Heading 2
\titleformat{\subsection}
  {\headingfont\bfseries\raggedright\fontsize{15pt}{16pt}\selectfont}{}{0.75em}{}
% Heading 3
\titleformat{\subsubsection}
  {\headingfont\bfseries\raggedright\fontsize{12pt}{14pt}\selectfont}{}{0.75em}{}

% Heading spacing
\titlespacing*{\section}{0pt}{7pt}{6pt}
\titlespacing*{\subsection}{0pt}{7pt}{6pt}
\titlespacing*{\subsubsection}{0pt}{7pt}{6pt}

\newcommand{\miniheading}[1]{%
  \par\noindent{\headingfont\bfseries\fontsize{12pt}{14pt}\selectfont #1}\par\vspace{4pt}%
}

% ---------- Paragraphing ----------
\setlength{\parindent}{0pt}
\setlength{\parskip}{6pt plus 1pt minus 1pt}

% ---------- Page numbers ----------
\usepackage{fancyhdr}
\pagestyle{fancy}
\fancyhf{}
\fancyhfoffset[R]{18pt}
\setlength{\footskip}{18pt}
\fancyfoot[R]{\sffamily\bfseries\footnotesize \thepage}
\renewcommand{\headrulewidth}{0pt}
\renewcommand{\footrulewidth}{0pt}

% INCOSE logo
\fancypagestyle{firstpage}{
  \fancyhf{}
  \fancyhfoffset[R]{18pt}
  \fancyhead[R]{%
    \smash{\raisebox{0pt}[0pt][0pt]{
      \begingroup\setlength{\fboxsep}{20pt}
        \colorbox{white}{\includegraphics[height=0.6in]{template-images/incose-logo.jpg}}%
      \endgroup
    }}
  }
  \fancyfoot[R]{\sffamily\bfseries\footnotesize \thepage}
  \renewcommand{\headrulewidth}{0pt}
  \renewcommand{\footrulewidth}{0pt}
}
% ---------- Title Formatting ----------
\makeatletter
\@ifundefined{theauthor}{}{\let\theauthor\relax}
\makeatother
\usepackage{titling}
\pretitle{\headingfont\bfseries\fontsize{24pt}{26pt}\selectfont\raggedright}
\posttitle{\par\vspace{-.3in}}
\preauthor{}\postauthor{}
\author{\mbox{}}
\date{}
\setlength{\droptitle}{-3.2\baselineskip}

% ---------- Author cards ----------
\newcommand{\authorcard}[5]{%
  {\headingfont\bfseries\fontsize{12pt}{14pt}\selectfont #1}\par
  {\headingfont\bfseries\fontsize{12pt}{14pt}\selectfont #2}\par
  {\headingfont\bfseries\fontsize{12pt}{14pt}\selectfont #3}\par
  {\headingfont\bfseries\fontsize{12pt}{14pt}\selectfont #4}\par
  {\headingfont\bfseries\fontsize{12pt}{14pt}\selectfont #5}\par
}

% ---------- Biography photo placeholder and entry ----------

\makeatletter
\newcommand{\authorpic}[1]{%
    \includegraphics[width=0.6in,height=0.6in,keepaspectratio,clip]{#1}%
}
\makeatother

\newcommand{\authorbioentry}[3]{%
  \noindent\begin{tabular}{@{}m{0.5in} M{\dimexpr\columnwidth-0.5in\relax}@{}}
    \authorpic{#1} & \textbf{#2}\par #3
  \end{tabular}\par\medskip
}

% ---------- Safe figure include ----------
\makeatletter
\newcommand{\colfig}[2][]{%
  \IfFileExists{#2}{\includegraphics[width=\linewidth,#1]{#2}}{%
    \fbox{\parbox[b][1.5in][c]{\linewidth}{\centering \textit{Missing figure: }#2}}}%
}
\makeatother

% ---------- References: APA via biblatex/biber ----------
\let\theauthor\relax
\usepackage{csquotes}
\usepackage[style=apa,backend=biber]{biblatex}
\addbibresource{references.bib}

% ---------- Highlight callouts ----------
\usepackage{changepage}
\newenvironment{highlight}[1][0.25in]{%
  \begin{adjustwidth}{#1}{#1}\itshape}{\end{adjustwidth}}

% ---------- Two-column setup ----------
\usepackage{multicol}
\setlength{\columnsep}{18pt}

% ---------- Hyperlinks ----------
\usepackage[hidelinks]{hyperref}

% =========================
% ===== Title & Authors ===
% =========================
\title{Optimized Human-Robot Co-Dispatch \\ Planning for Petro-Site Surveillance \\ under Varying Criticalities}

\begin{document}
\maketitle
\thispagestyle{firstpage}

% ---- Authors ----
% ---- For the initial paper submission, do not include any author information. For the final paper submission, format author information as shown below. ------------------

\noindent
\begin{tabular*}{\textwidth}{@{\extracolsep{\fill}} A{0.32\textwidth} A{0.32\textwidth} A{0.32\textwidth}}
  \authorcard{Nur Ahmad Khatim}{Institute Technology of Sepuluh Nopember}{Surabaya, Indonesia}{}{} &
  % \authorcard{Author Two}{Organization}{Street Address}{City, Province, Postal}{author.two@email.com} &
  % \authorcard{Author Three}{Organization}{Street Address}{City, Province, Postal}{author.three@email.com} \\
  % \multicolumn{3}{@{}c@{}}{\rule{0pt}{0.9\baselineskip}} \\[-0.2\baselineskip]
  % \authorcard{Author Four}{Organization}{Street Address}{City, Province, Postal}{author.four@email.com} &
  % \authorcard{Author Five}{Organization}{Street Address}{City, Province, Postal}{author.five@email.com} &
  % \authorcard{Yan Akhra Pratama}{Institute Technology of Sepuluh Nopember}{Surabaya, Indonesia}{}{pratamaakhra@gmail.com} &
 \authorcard{Mansur M. Arief}{King Fahd University of Petroleum and Minerals}{Dhahran, Saudi Arabia}{}{} \\
\end{tabular*}
% \addvspace{.75in}

% ---- Two columns begin immediately after authors ----
\begin{multicols*}{2}
\raggedcolumns

% ---- Copyright ----
{\headingfont\bfseries\fontsize{8pt}{12pt}\selectfont
Copyright~\textcopyright~ \the\year{} by the author(s). Permission granted to INCOSE to publish and use.}
\\
% =========================
% ===== Abstract/Keywords =
% =========================
\phantomsection
\miniheading{Abstract}
Securing petroleum infrastructure requires balancing autonomous system efficiency with human judgment for threat escalation, which is a challenge unaddressed by classical facility location models that assume homogeneous resources and uniform service standards. This paper formulates the Human-Robot Co-Dispatch Facility Location Problem (HRCD-FLP), a capacitated facility location variant that explicitly incorporates: tiered infrastructure criticality with differentiated service level agreements, human-robot co-dispatch with supervision ratio constraints, and minimum utilization requirements for command centers. We develop a hybrid solution approach in a case study across three technology maturity scenarios, which demonstrates that transitioning from conservative (1:3 human-robot supervision) to future autonomous operations (1:10 supervision) yields significant cost reduction while maintaining complete coverage for critical infrastructure surveillance and security services.
\phantomsection
\subsubsection{Keywords}
Human-robot teaming, facility location problem, critical infrastructure protection

% =========================
% ===== Main Content ======
% =========================
\section{Introduction}

The global petroleum infrastructure represents one of the most extensive critical infrastructure networks, with over 2.7 million miles of pipelines in the United States alone and similar networks spanning the Arabian Peninsula and other major hydrocarbon basins (\cite{coburn2020oil, klass2014transporting}). These networks exhibit a hierarchical topology: high-criticality nodes (refineries, processing facilities, storage terminals) interconnected by extensive linear assets of varying criticality. Protection of such infrastructure has emerged as a paramount concern for national security and economic stability (\cite{pashchenko2024main, bajpai2007securing}), particularly as threat vectors now encompass both physical intrusion and cyber-physical attacks (\cite{mohammed2022cybersecurity}).

The advent of Fourth Industrial Revolution (4IR) technologies has catalyzed a paradigm shift in industrial security operations. Unmanned aerial vehicles (UAVs), autonomous ground vehicles (AGVs), and integrated sensor networks now complement human surveillance capacity. However, integrating these heterogeneous assets presents a complex systems engineering challenge---not merely technological, but fundamentally one of resource allocation, facility location, and human-machine teaming under operational constraints. The DARPA Subterranean Challenge demonstrated how multi-robot systems deployed without capability-aware coordination can lead to mission failures (\cite{agha2021nebula}).

Traditionally, the facility location problem (FLP) and services coverage are modeled as covering problems in operations research, in which classical models minimize facility costs while covering demand points, or maximize demands served given available servers (\cite{li2011covering}). Yet existing FLP models remain inadequate for modern petroleum security: they treat resources as homogeneous units, ignoring the distinction between autonomous systems capable of continuous patrol and human operators essential for ethical escalation decisions (\cite{adams2024security}). Conventional covering models impose uniform service standards despite stark asymmetries (i.e. protecting a Tier-1 refinery where breach enables catastrophic sabotage differs fundamentally from monitoring remote pipeline segments) (\cite{cisa2024plan}). With ransomware attacks on oil and gas surging 935\% between 2024--2025 (\cite{zscaler2025ransomware}), the absence of human-in-the-loop constraints creates a dangerous blind spot as organizations deploy increasingly autonomous fleets (\cite{adams2024hrt, degruyter2025hitl}).

We address these gaps by proposing conceptual model bridging the hierarchical criticality structure of petroleum infrastructure with human-in-the-loop supervision constraints that reflect both regulatory requirements for autonomous system deployment and their maturity in security-critical applications (\cite{degruyter2025hitl, cisa2024plan}). To that end, we consider: (i) tiered infrastructure criticality with differentiated service level agreements (\cite{church2004identifying}); (ii) human-robot co-dispatch with supervision ratio constraints (\cite{hagenow2023coordinated}); (iii) redundant coverage calibrated to asset vulnerability (\cite{hogan1986backup}); and (iv) command center capacity constraints (\cite{daskin2013network}). We then build a mathematical model formulation \textit{Human-Robot Co-Dispatch Facility Location Problem} (HRCD-FLP) to bridge classical facility location with human-robot teaming for critical infrastructure protection. We then demonstrate applicability through a Dhahran district case study, benchmarking against baseline models across technology maturity scenarios.

\section{Literature Review}

This work bridges three research domains: critical infrastructure protection (CIP), facility location problems (FLPs), and human-robot collaboration (HRC) for surveillance.

\subsection{Critical Infrastructure Protection}

Petroleum infrastructure protection has received sustained attention since 2001. TSA Pipeline Security Guidelines and DHS frameworks classify facilities based on criticality---target viability, energy supply importance, and weaponization potential (\cite{bajpai2007securing, pashchenko2024main}). Church and Scaparra's $r$-interdiction models identify infrastructure elements whose loss maximally degrades system performance, establishing that strategic fortification must account for attacker-defender dynamics \cite{church2004identifying, scaparra2008bilevel}. Contemporary security increasingly leverages autonomous systems; Saudi Aramco has deployed drones reducing inspection times by 90\% at facilities like Uthmaniyah Gas Plant \cite{alwalaie2021aramco}. However, optimization frameworks for strategic placement of hybrid human-robot dispatch centers remain absent.

\subsection{Facility Location Problems}

The FLP originates from Weber's industrial location theory and Hakimi's switching center placement \cite{hakimi1964optimum}. The Set Covering Location Problem (SCLP) seeks minimum facilities ensuring demand coverage within service thresholds \cite{toregas1971location}, while the Maximal Covering Location Problem (MCLP) maximizes covered demand given fixed facilities \cite{church1974maximal}. The Capacitated Facility Location Problem (CFLP) adds per-facility demand constraints \cite{daskin2013network}. Extensions include backup coverage \cite{hogan1986backup} and bilevel interdiction-fortification models \cite{scaparra2008bilevel}. Emergency medical services literature has developed tiered response systems with different vehicle types serving different priorities \cite{farahani2012covering}---conceptually parallel to human-robot teaming---but heterogeneous resource integration with supervision requirements remains unaddressed.

\subsection{Human-Robot Collaboration}

Autonomous security robots now offer 24/7 patrolling with thermal imaging across critical infrastructure in over 15 countries \cite{adams2024security}. Despite capabilities, ethical and regulatory considerations require human oversight for escalation decisions, manifesting as supervision ratio constraints \cite{chen2011supervisory}. The human-robot teaming literature identifies operator multitasking, trust calibration, and cognitive workload as key collaboration factors \cite{chen2011supervisory, haring2021understanding}. Operations research has addressed human-robot task allocation in warehouse environments \cite{boysen2019warehouse}, but strategic facility location for security teams remains unexamined.

% \subsection{Research Gaps}
Three gaps emerge at these domain intersections. First, CIP literature lacks optimization models for hybrid human-robot workforces. Second, FLP literature has not incorporated human-robot constraints including supervision ratios and differential efficiency rates. Third, HRC literature focuses on operational coordination with limited attention to strategic facility location. This paper addresses these gaps through an integrated HRCD-FLP formulation combining tiered criticality, capacitated facility location with SLA constraints, explicit human-robot modeling, supervision constraints, and minimum utilization requirements.

\section{System Overview}

% \subsection{Operational Context}

% Saudi Aramco manages a vast critical infrastructure network spanning the Eastern Province of Saudi Arabia. In alignment with Saudi Vision 2030, the company is transitioning toward smart security operations leveraging autonomous systems. This transition requires a strategic decision-support framework optimizing hybrid human-robot security team deployment.

The system-of-interest comprises interacting elements. First, \textit{demand sites} represent petroleum assets requiring surveillance coverage quantified in Surveillance Coverage Units (SCU), encompassing high-criticality concentrated facilities (refineries, processing plants, storage terminals) and distributed linear assets (pipeline segments, valve stations). Second, \textit{command centers} are candidate facility locations from which security resources are dispatched, each characterized by construction costs, operational overhead, and physical capacity constraints. Third, \textit{security resources} constitute a heterogeneous fleet of human personnel and autonomous robotic units with different capabilities, costs, and constraints.

\subsection{Stakeholder Requirements}
The design is driven by five key stakeholder requirements. \textit{Coverage completeness} mandates that all demand sites receive 100\% of required security coverage without exception. \textit{Response time SLAs} require that command centers may only serve demand sites within maximum allowable response distances, with thresholds differentiated by asset criticality tier. \textit{Human-in-the-loop constraints} stipulate that autonomous systems must operate under human supervision at prescribed ratios (e.g., 1:5 human-to-robot) to ensure ethical decision-making capacity for threat escalation scenarios. \textit{Economic efficiency} requires minimizing total infrastructure and operational costs subject to coverage and supervision constraints. Finally, \textit{infrastructure utilization thresholds} ensure that opened facilities achieve minimum occupancy rates to justify capital investment and prevent proliferation of underutilized command centers.

\subsection{System Architecture}
The key aspects of the model includes heterogeneous resources, encompassing three hierarchical decision layers. At the \textit{strategic layer}, the model determines optimal locations among candidate sites and selects operational levels (High, Medium, Low) for each command center, where higher levels provide greater capacity and faster response capabilities at increased fixed cost. The \textit{tactical layer} assigns each demand site to exactly one command center through a single-sourcing constraint, ensuring unified command responsibility and SLA compliance while simplifying operational coordination. At the \textit{operational layer}, the model determines the resource mix of humans and robots at each facility, satisfying aggregate coverage requirements, physical capacity limits, and supervision ratio constraints.

The proposed system design involves fundamental trade-offs. The coverage-cost trade-off reflects tension between deploying more facilities (ensuring tighter coverage but increasing fixed costs) versus fewer facilities (reducing capital expenditure but risking SLA violations for remote assets). The human-robot resource trade-off balances robots' higher efficiency (3--5$\times$ human patrol capacity) and lower marginal cost against their requirement for human supervision and inability to exercise judgment for threat escalation. Facility level selection weighs premium construction and operational costs of higher-level facilities against their faster response times and greater capacity. Finally, the centralization-distribution trade-off considers whether consolidated high-capacity centers achieving economies of scale outperform distributed lower-capacity outposts that reduce response distances to remote assets.

% The HRCD-FLP formulation captures these trade-offs through an objective function minimizing total cost subject to constraints encoding stakeholder requirements. The following section presents the formal mathematical specification.


\begin{figure*}
\centering
\begin{tikzpicture}[
    font=\sffamily\footnotesize,
    node distance=10mm and 12mm,
    block/.style={rectangle, draw=black, rounded corners, align=center,
                  minimum width=2.6cm, minimum height=0.9cm},
    bigblock/.style={rectangle, draw=black, rounded corners, align=center,
                     minimum width=3.2cm, minimum height=1.1cm, fill=gray!10},
    dashedblock/.style={rectangle, draw=black, dashed, rounded corners,
                        align=center, minimum width=3.0cm, minimum height=0.9cm},
    arrow/.style={-{Stealth[scale=0.8]}, thick},
    dashedarrow/.style={-{Stealth[scale=0.8]}, thick, dashed}
]

% ---------------- Left column: Physical world ----------------
\node[block] (assets) {Petroleum Assets\\[-1pt]
{\scriptsize Refineries, Pipelines}\\[-2pt]
{\tiny }};

\node[block, below=of assets] (demand) {Surveillance Demand\\[-1pt]
{\scriptsize (SCU$_j$)}\\[-1pt]
{\scriptsize}};

\draw[arrow] (assets) -- (demand);

% ---------------- Middle-left: Policy & constraints ----------------
\node[dashedblock, right=of assets, yshift=-5mm] (policy) {Regulatory Constraints\\[-1pt]
{\scriptsize Human-in-the-Loop}\\[-1pt]
{\scriptsize Supervision Ratios}};

\node[dashedblock, below=of policy] (sla) {Service Level Agreements\\[-1pt]
{\scriptsize Response Time Limits}\\[-1pt]
{\scriptsize by Criticality Tier}};

% ---------------- Middle: Optimization core ----------------
\node[bigblock, right=of demand, xshift=8mm, yshift=8mm] (model) {
\textbf{HRCD-FLP}\\[-1pt]
{\scriptsize Strategic: Location \& Level}\\[-1pt]
{\scriptsize Tactical: Assignment}\\[-1pt]
{\scriptsize Operational: Resource Mix}
};

\draw[arrow] (demand) -- (model);
\draw[dashedarrow] (policy) -- (model);
\draw[dashedarrow] (sla) -- (model);

% ---------------- Right: Resources ----------------
\node[block, right=of model, yshift=10mm] (robots) {Autonomous Units\\[-1pt]
{\scriptsize UAVs, UGVs}\\[-1pt]
{\scriptsize High Efficiency}};

\node[block, below=of robots] (humans) {Human Supervisors\\[-1pt]
{\scriptsize Judgment \& Escalation}\\[-1pt]
{\scriptsize Ethical Control}};

\draw[arrow] (model) -- (robots);
\draw[arrow] (model) -- (humans);

% ---------------- Bottom: Command centers ----------------
\node[bigblock, below=of model, yshift=-2mm] (centers) {Command Centers\\[-1pt]
{\scriptsize Capacity Constraints}\\[-1pt]
{\scriptsize Multi-Level (H/M/L)}};

\draw[arrow] (model) -- (centers);
\draw[arrow] (centers.east) -- ++(0.5,0) |- (robots.south);
\draw[arrow] (centers.east) -- ++(0.3,0) |- (humans.south);

% ---------------- Supervision link ----------------
\draw[dashedarrow] (humans.east) -- ++(0.4,0) |- node[right, pos=0.25, font=\scriptsize] {supervises} (robots.east);

\end{tikzpicture}
\caption{Conceptual architecture of the HRCD-FLP framework. 
% Tiered petroleum assets generate criticality-weighted surveillance demand, optimized under SLA and supervision constraints to determine command center locations and human-robot resource allocation.
}
\label{fig:conceptual_model}
\end{figure*}

\section{Model Formulation}

This section presents the mathematical specification of the model. We formulate the problem as a multi-level capacitated facility location model with heterogeneous resources, capturing the key trade-offs identified in the preceding system analysis.

\subsection{Problem Setting}

Consider a set of candidate command center locations $I = \{1, \ldots, m\}$ and a set of demand sites $J = \{1, \ldots, n\}$ requiring security surveillance. Each candidate location $i \in I$ may be developed into a command center at one of three operational levels $l \in L = \{\text{High}, \text{Medium}, \text{Low}\}$, where each level is characterized by distinct cost structures, resource capacities, and response time capabilities reflecting infrastructure quality and technological sophistication.

Demand sites require security coverage measured in Surveillance Coverage Units (SCU), a composite metric capturing patrol frequency, sensor density, and response readiness appropriate to asset criticality. This demand is satisfied by deploying two resource types $k \in K = \{\text{Robot}, \text{Human}\}$ from assigned command centers. Robots provide high-efficiency continuous patrol capability, while humans contribute supervisory oversight and contextual judgment for threat escalation. The resource mix at each facility is governed by both site-specific coverage requirements and global supervision policies ensuring human-in-the-loop control.

\subsection{Parameters}

The model is parameterized by cost, capacity, and demand data. For costs, let $F_{il}$ denote the fixed construction and operational overhead for establishing facility $i$ at level $l$, and let $C_{ik}$ represent the unit deployment cost for resource type $k$ at location $i$, capturing salary and benefits for human personnel or amortized capital and maintenance costs for robotic units.

Capacity parameters include $\text{MAXCAP}_{lk}$, the maximum number of resource type $k$ that a level-$l$ facility can accommodate, and $\text{MINCAP}_{lk}$, the minimum resource deployment required to maintain operational readiness if a facility is opened. Response capability is captured by $t_{ijl}$, the response time from facility $i$ at level $l$ to demand site $j$, where higher-level facilities may deploy faster assets (e.g., VTOL drones versus standard quadcopters). Each demand site $j$ has an associated service level agreement $S_j$ specifying maximum allowable response time, differentiated by asset criticality tier.

Demand parameters include $D_j$, the total surveillance requirement (in SCU) at site $j$, and $\alpha_j$, the site-specific human-robot mix ratio reflecting task complexity and escalation likelihood. The global supervision ratio $\alpha$ encodes the minimum human-to-robot ratio mandated by policy, ensuring adequate oversight as autonomous fleet sizes scale.

\subsection{Decision Variables}

The model determines three classes of decisions corresponding to the strategic, tactical, and operational layers identified in the system architecture. The binary variable $x_{il} \in \{0, 1\}$ equals 1 if candidate location $i$ is developed at operational level $l$, and 0 otherwise, representing the strategic facility location and level selection. The binary variable $y_{ij} \in \{0, 1\}$ equals 1 if demand site $j$ is assigned to facility $i$ for service, encoding the tactical demand-to-facility assignment. Finally, the integer variable $z_{ik} \in \mathbb{Z}^+$ specifies the number of resource type $k$ deployed at facility $i$, determining the operational resource mix.

\subsection{Objective Function}

The objective minimizes total system cost, comprising fixed infrastructure costs and variable resource deployment costs:
\begin{equation}
\min_{x, y, z} \quad Z = \sum_{i \in I} \sum_{l \in L} F_{il} \, x_{il} + \sum_{i \in I} \sum_{k \in K} C_{ik} \, z_{ik}.
\label{eq:objective}
\end{equation}

\noindent The first term aggregates construction and operational overhead across all opened facilities, while the second term captures ongoing personnel and equipment costs scaled by deployment quantities.

\subsection{Constraints}

The optimization is subject to constraints encoding stakeholder requirements and operational feasibility.

\paragraph{Facility Configuration} Each candidate location may host at most one facility, and if developed, must be assigned exactly one operational level:
\begin{equation}
\sum_{l \in L} x_{il} \leq 1, \quad \forall i \in I.
\label{eq:single_level}
\end{equation}

\paragraph{Demand Coverage} Every demand site must be assigned to at least one command center:
\begin{equation}
\sum_{i \in I} y_{ij} \geq 1, \quad \forall j \in J.
\label{eq:demand_assignment}
\end{equation}
\noindent While the inequality permits redundant assignment for backup coverage at critical sites, practical implementations typically achieve single-sourcing to ensure unified command responsibility.

\paragraph{Assignment Feasibility} A demand site may only be assigned to an active facility, which linking tactical assignments to strategic location decisions:
\begin{equation}
y_{ij} \leq \sum_{l \in L} x_{il}, \quad \forall i \in I, \; \forall j \in J.
\label{eq:logical_link}
\end{equation}

\paragraph{Service Level Compliance} Assignments must respect response time constraints, where a facility at level $l$ may serve site $j$ only if the response time falls within the SLA threshold. Using big-$M$ formulation, we have
\begin{equation}
t_{ijl} \, x_{il} \leq S_j + M(1 - y_{ij}), \quad \forall i \in I, \; \forall j \in J, \; \forall l \in L.
\label{eq:sla}
\end{equation}
\noindent This constraint ensures that high-criticality Tier-1 assets with stringent SLAs can only be served by nearby facilities or those equipped with rapid-deployment capabilities.

\paragraph{Resource Capacity Bounds} Resource deployment at each facility must respect physical capacity limits determined by the selected operational level:
\begin{equation}
z_{ik} \leq \sum_{l \in L} \text{MAXCAP}_{lk} \, x_{il}, \quad \forall i \in I, \; \forall k \in K.
\label{eq:max_cap}
\end{equation}
\noindent The upper bound prevents over-allocation beyond infrastructure capacity. The lower bound ensures minimum staffing for operational readiness when a facility is opened, preventing proliferation of skeleton crews:
\begin{equation}
z_{ik} \geq \sum_{l \in L} \text{MINCAP}_{lk} \, x_{il}, \quad \forall i \in I, \; \forall k \in K.
\label{eq:min_cap}
\end{equation}

\paragraph{Coverage Satisfaction} Deployed resources must be sufficient to meet the aggregate surveillance demand of all assigned sites, accounting for the site-specific human-robot task allocation. Robot deployment must satisfy:
\begin{equation}
z_{i,\text{Robot}} \geq \sum_{j \in J} \frac{D_j}{1 + \alpha_j} \, y_{ij}, \quad \forall i \in I.
\label{eq:robot_coverage}
\end{equation}
\noindent Meanwhile, human deployment must satisfy:
\begin{equation}
z_{i,\text{Human}} \geq \sum_{j \in J} \frac{D_j \, \alpha_j}{1 + \alpha_j} \, y_{ij}, \quad \forall i \in I .
\label{eq:human_coverage}
\end{equation}
\noindent Here, $\alpha_j$ governs the demand split: sites with higher $\alpha_j$ require proportionally more human involvement due to task complexity or escalation sensitivity.

\paragraph{Human-in-the-Loop Supervision}
A global policy constraint ensures adequate human oversight for autonomous operations, requiring the human contingent to scale with robot fleet size:
\begin{equation}
z_{i,\text{Human}} \geq \alpha \cdot z_{i,\text{Robot}}, \quad \forall i \in I.
\label{eq:supervision}
\end{equation}
\noindent This constraint operationalizes the human-in-the-loop requirement central to this work. As organizations deploy larger autonomous fleets, the supervision ratio $\alpha$ (e.g., 1:5 indicating one human per five robots) mandates proportional human presence for ethical decision-making and regulatory compliance. The ratio $\alpha$ serves as a policy lever reflecting technology maturity: conservative deployments with less proven autonomy require higher ratios, while mature systems with established trust may operate under relaxed supervision.

\subsection{Model Characteristics}

The HRCD-FLP formulation constitutes a mixed-integer linear program (MILP) with $|I| \cdot |L| + |I| \cdot |J|$ binary variables and $|I| \cdot |K|$ integer variables. The problem generalizes the classical capacitated facility location problem through three extensions: multi-level facility selection introducing discrete infrastructure tiers, heterogeneous resources with differential efficiency and supervision requirements, and the supervision ratio constraint \eqref{eq:supervision} coupling human and robot deployment decisions. These extensions increase model expressiveness at the cost of computational complexity, motivating the solution approach presented in the following section.


% \section{Model Formulation}
% We consider a set of candidate locations $I$ and a set of demand sites $J$. Each candidate location can be developed into a command center with a specific operational level $l \in L = \{\text{High}, \text{Medium}, \text{Low}\}$. Each level is characterized by distinct cost structures, resource capacities, and response time capabilities. 

% Demand sites require security coverage measured in Surveillance Coverage Units (SCU). This demand is satisfied by a mix of two resource types $k \in K = \{\text{Robot}, \text{Human}\}$. The mix is governed by site-specific characteristics (e.g., complexity requiring more human intervention) and global supervision policies.

% \subsubsection{Sets and Indices}
% \begin{itemize}
%     \item $I$: Set of candidate command center locations, indexed by $i$.
%     \item $J$: Set of demand sites, indexed by $j$.
%     \item $L$: Set of facility levels (High, Medium, Low), indexed by $l$.
%     \item $K$: Set of resource types (Robot, Human), indexed by $k$.
% \end{itemize}

% \subsubsection{Parameters}
% \begin{itemize}
%     \item \textbf{Costs:}
%     \begin{itemize}
%         \item $F_{il}$: Fixed construction and overhead cost for facility $i$ at level $l$.
%         \item $C_{ik}$: Unit deployment cost for resource $k$ at location $i$.
%     \end{itemize}
%     \item \textbf{Capacities and Capabilities:}
%     \begin{itemize}
%         \item $MAXCAP_{lk}$: Maximum capacity of resource $k$ for a facility at level $l$.
%         \item $MINCAP_{lk}$: Minimum required resource $k$ for a facility at level $l$ (if built).
%         \item $t_{ijl}$: Response time from facility $i$ at level $l$ to site $j$. Higher levels may deploy faster assets (e.g., VTOL drones vs standard quadcopters).
%         \item $S_j$: Service Level Agreement (maximum allowable response time) for site $j$.
%     \end{itemize}
%     \item \textbf{Demand:}
%     \begin{itemize}
%         \item $D_j$: Total security demand (SCU) at site $j$.
%         \item $\alpha_j$: Site-specific mix ratio required to cover demand (Human/Robot balance).
%         \item $\alpha$: Global supervision ratio (minimum Humans per Robot).
%     \end{itemize}
% \end{itemize}

% \subsubsection{Decision Variables}
% \begin{itemize}
%     \item $x_{il} \in \{0, 1\}$: 1 if candidate location $i$ is developed at level $l$, 0 otherwise.
%     \item $y_{ij} \in \{0, 1\}$: 1 if demand site $j$ is assigned to facility $i$, 0 otherwise.
%     \item $z_{ik} \in \mathbb{Z}^+$: Number of resources of type $k$ deployed at facility $i$.
% \end{itemize}

% \subsubsection{Objective Function}
% Minimize the total monthly cost ($Z$), comprising fixed facility costs and variable resource deployment costs:
% \begin{equation}
% \text{Minimize } Z = \sum_{i \in I} \sum_{l \in L} F_{il} x_{il} + \sum_{i \in I} \sum_{k \in K} C_{ik} z_{ik}
% \end{equation}

% \subsubsection{Constraints}
% \begin{enumerate}
%     \item \textbf{Single Level Selection:} Each candidate location can strictly accommodate at most one facility type.
%     \begin{equation}
%     \sum_{l \in L} x_{il} \le 1, \quad \forall i \in I
%     \end{equation}
    
%     \item \textbf{Demand Assignment:} Every demand site must be assigned to exactly one facility.
%     \begin{equation}
%     \sum_{i \in I} y_{ij} \ge 1, \quad \forall j \in J
%     \end{equation}
    
%     \item \textbf{Logical Link:} Assignments can only be made to active facilities.
%     \begin{equation}
%     y_{ij} \le \sum_{l \in L} x_{il}, \quad \forall i \in I, j \in J
%     \end{equation}
    
%     \item \textbf{SLA Compliance:} A facility at level $l$ can only serve site $j$ if the response time is within the limit.
%     \begin{equation}
%     t_{ijl} x_{il} \le S_j + M(1 - y_{ij}), \quad \forall i, j, l
%     \end{equation}
    
%     \item \textbf{Resource Capacity (Upper Bound):} Total resources must not exceed the physical capacity of the chosen level.
%     \begin{equation}
%     z_{ik} \le \sum_{l \in L} MAXCAP_{lk} x_{il}, \quad \forall i \in I, k \in K
%     \end{equation}
    
%     \item \textbf{Minimum Resource Requirement (Lower Bound):} Opened facilities must maintain a base level of readiness.
%     \begin{equation}
%     z_{ik} \ge \sum_{l \in L} MINCAP_{lk} x_{il}, \quad \forall i \in I, k \in K
%     \end{equation}
    
%     \item \textbf{Coverage Satisfaction:} Resources must be sufficient to cover the split demand (Robotic vs. Human tasks) of all assigned sites.
%     \begin{align}
%     z_{i, \text{Robot}} &\ge \sum_{j \in J} \frac{D_j}{1+\alpha_j} y_{ij}, \quad \forall i \in I \\
%     z_{i, \text{Human}} &\ge \sum_{j \in J} \frac{D_j \alpha_j}{1+\alpha_j} y_{ij}, \quad \forall i \in I
%     \end{align}
    
%     \item \textbf{Global Supervision:} A global policy constraint ensuring adequate human oversight for robotic fleets.
%     \begin{equation}
%     z_{i, \text{Human}} \ge \alpha \cdot z_{i, \text{Robot}}, \quad \forall i \in I
%     \end{equation}
% \end{enumerate}

\section{Method}
The optimization problem posed by the HRCD-FLP is NP-hard, involving complex interdependencies between facility location, level selection, and resource allocation. To address this computational challenge, we employed a hybrid solution strategy that balances optimality with scalability. For smaller instances and model validation, we utilized an exact solution approach implemented via the Gurobi Optimizer, which guarantees global optimality through a branch-and-bound mechanism. This exact solver serves as a baseline for benchmarking performance. Recognizing the computational limits of exact methods for large-scale deployments, we further developed a specialized two-stage metaheuristic. The first stage consists of a multi-level constructive greedy algorithm that iteratively assigns demand sites to facilities by minimizing marginal costs while dynamically adjusting facility levels to meet SLA and capacity constraints. The second stage refines this initial solution through a "Best Improvement" local search, which explores the solution space using shift, swap, drop, and open moves to escape local optima and enhance solution quality.

\section{Data Collection and Generation}
The study utilizes a hybrid dataset designed to simulate the Dhahran Core Area, combining real-world geospatial topology with parameter synthesis based on industrial security standards. The experimental environment comprises 15 candidate locations and 50 demand sites, distributed to mimic pipeline corridors and scattered high-value assets. Geodesic distances were calculated to ensure realistic response time modeling. Facility capabilities were modeled across three tiers: High-level facilities represent premium infrastructure with 1.5x base fixed costs but high capacity ($\approx 240$ robots) and rapid response assets; Medium-level facilities reflect standard infrastructure; and Low-level facilities act as satellite outposts with lower costs but limited capacity and slower response times.

To evaluate the impact of technological maturity, we defined three scenarios: Conservative, Balanced, and Future. These scenarios vary the global supervision ratio ($\alpha$), the robot cost multiplier, and the site-specific mix scaler. The Conservative scenario represents current operational constraints with a 1:3 human-to-robot supervision ratio and baseline robot costs. The Balanced scenario reflects near-term improvements with a 1:5 ratio and 10\% lower robot costs. The Future scenario projects a mature autonomous ecosystem with a 1:10 ratio and 20\% cost reduction, allowing for highly automated operations.

\section{Results}
This section presents the results of the optimization model across the defined scenarios, analyzing the cost implications, resource allocations, and network configurations.

\subsection{Optimization Performance}
The experimental evaluation demonstrates the efficacy of the proposed model across the three defined technological scenarios. As summarized in Table \ref{tab:experiment_results}, the Exact method consistently identified optimal solutions with significantly lower costs compared to the Heuristic approach, although the Heuristic demonstrated competitive performance with optimality gaps averaging around 6\%. Notably, the exact solver proved highly efficient for the test instances, with solution times under 1.5 seconds, whereas the heuristic, while robust, required slightly more time due to its iterative search process.

\begin{table*}[t]
\centering
\begingroup
\renewcommand{\arraystretch}{2}
\begin{tabularx}{\textwidth}{@{}P{0.15\textwidth} P{0.1\textwidth} Y Y Y Y Y Y@{}}
    \rowcolor{tableheader}
    \multicolumn{1}{c}{\headingfont\bfseries Scenario} &
    \multicolumn{1}{c}{\headingfont\bfseries Method} &
    \multicolumn{1}{c}{\headingfont\bfseries Facilities} &
    \multicolumn{1}{c}{\headingfont\bfseries Robots} &
    \multicolumn{1}{c}{\headingfont\bfseries Humans} &
    \multicolumn{1}{c}{\headingfont\bfseries Cost (\$)} &
    \multicolumn{1}{c}{\headingfont\bfseries Time (s)} &
    \multicolumn{1}{c}{\headingfont\bfseries Gap (\%)} \\
    \addlinespace[4pt]
    Conservative & Exact & 4 & 207 & 149 & 692,450 & 1.32 & -- \\
     & Heuristic & 9 & 212 & 149 & 734,700 & 6.70 & 6.10 \\
    \addlinespace[4pt]
    Balanced & Exact & 4 & 229 & 127 & 610,175 & 0.25 & -- \\
     & Heuristic & 9 & 234 & 127 & 652,185 & 6.03 & 6.88 \\
    \addlinespace[4pt]
    Future & Exact & 3 & 256 & 98 & 508,000 & 0.26 & -- \\
     & Heuristic & 8 & 261 & 99 & 537,820 & 6.51 & 5.87 \\
\end{tabularx}
\caption{Experiment Results Comparison}
\label{tab:experiment_results}
\endgroup
\end{table*}

\subsection{Resource Allocation Analysis}
Analyzing the resource allocation reveals distinct trends driven by the supervision constraints and cost parameters. As illustrated in Figure \ref{fig:cc_levels}, the distribution of command center levels shifts across scenarios. In the Conservative scenario, the system favors a distributed network of High-level facilities to accommodate the large human workforce required by the 1:3 supervision ratio. Conversely, as the supervision ratio relaxes in the Future scenario, the model consolidates operations into a leaner network of facilities, reflecting the improved economics of automation.

\begin{figure*}
    \centering
    \includegraphics[width=\linewidth]{figures/command_center_levels.pdf}
    \caption{Distribution of Command Center Levels across Scenarios}
    \label{fig:cc_levels}
\end{figure*}

Figures \ref{fig:res_exact} and \ref{fig:res_heuristic} detail the specific resource counts for the Exact and Heuristic solutions, respectively. The Exact method (Figure \ref{fig:res_exact}) achieves a tighter alignment of resources to demand, resulting in lower total counts of both robots and humans compared to the Heuristic method (Figure \ref{fig:res_heuristic}). This efficiency underscores the value of global optimization in minimizing capital and operational expenditures. The comparison highlights that while the heuristic provides feasible solutions, the exact method extracts approximately 15-20\% more efficiency from the available resources.

\begin{figure*}
    \centering
    \includegraphics[width=\linewidth]{figures/facility_resources_exact.pdf}
    \caption{Facility Resource Allocation (Exact Method)}
    \label{fig:res_exact}
\end{figure*}

\begin{figure*}
    \centering
    \includegraphics[width=\linewidth]{figures/facility_resources_heuristic.pdf}
    \caption{Facility Resource Allocation (Heuristic Method)}
    \label{fig:res_heuristic}
\end{figure*}

\subsection{Demand Assignment Analysis}
The spatial distribution of demand assignments, as visualized in Figures \ref{fig:result_conservative}, \ref{fig:result_balanced}, and \ref{fig:result_future}, provides further insight into the network topology. The demand assignment analysis shows how the model adapts to the changing constraints. In the Conservative scenario, the dense network of facilities ensures that no demand site is too far from a command center, critical for maintaining the high supervision ratios. As we move to the Balanced and Future scenarios, the "catchment areas" of each facility expand, taking advantage of the reduced need for human proximity and the high efficiency of the autonomous units. This consolidation allows for a more streamlined and cost-effective security architecture.

\begin{figure*}
    \centering
    \includegraphics[width=\linewidth]{figures/result_conservative_combined.pdf}
    \caption{Conservative Scenario Breakdown}
    \label{fig:result_conservative}
\end{figure*}

\begin{figure*}
    \centering
    \includegraphics[width=\linewidth]{figures/result_balanced_combined.pdf}
    \caption{Balanced Scenario Breakdown}
    \label{fig:result_balanced}
\end{figure*}

\begin{figure*}
    \centering
    \includegraphics[width=\linewidth]{figures/result_future_combined.pdf}
    \caption{Future Scenario Breakdown}
    \label{fig:result_future}
\end{figure*}

\subsection{Executable Deployment Plan}
Based on the Balanced scenario, we propose a three-phase deployment plan for the Dhahran district. Phase 1 (Months 1-6) focuses on infrastructure, involving the construction of command centers at optimal locations, prioritizing peripheral sites that serve the majority of demand. Phase 2 (Months 7-12) initiates resource deployment, installing robotic units and training human supervisors at the mandated 1:5 ratio. Phase 3 (Years 2-5) manages the technology migration, gradually transitioning towards the Future scenario by consolidating facilities and capitalizing on increased automation to reduce monthly operational costs.

\subsection{Economic Impact Analysis}
Comparing the optimized solutions against a baseline of full human deployment reveals substantial savings. The Conservative scenario yields an estimated 25\% reduction in costs by introducing robotic augmentation. As technology matures, the Balanced and Future scenarios offer dramatic potential savings of 45\% and 65\% respectively. These projections highlight the transformative economic potential of integrating autonomous systems into security operations, provided that the requisite regulatory frameworks for reduced supervision are in place.

\section{Conclusions}
This study formulated and solved a Capacitated Facility Location Problem tailored for security command center optimization at Saudi Aramco's Dhahran headquarters. The mathematical model successfully integrates multiple real-world constraints including SLA compliance, human-robot supervision ratios, facility capacity limits, and minimum utilization requirements.

\subsection{Key Contributions}
\begin{enumerate}
    \item \textbf{Novel Problem Formulation:} We adapted the classical CFLP framework to explicitly model the trade-off between human security personnel and autonomous robotic units, incorporating supervision constraints unique to industrial security applications.
    
    \item \textbf{Computational Framework:} We developed a hybrid solution approach combining exact optimization (Gurobi) for benchmark solutions with a constructive greedy heuristic enhanced by Shift, Swap, and Drop/Open local search moves. The heuristic achieves less than 2.5\% optimality gap while running 10x faster.
    
    \item \textbf{Scenario Analysis:} By varying technological maturity parameters while keeping geography constant, we quantified the potential cost savings from automation: up to 63.4\% reduction when transitioning from conservative to fully autonomous operations.
    
    \item \textbf{Deployment Roadmap:} The executable deployment plan provides Saudi Aramco with a phased approach to implement the optimal solution, starting with current technology and evolving toward higher automation as AI capabilities mature.
\end{enumerate}

\subsection{Limitations}
\begin{itemize}
    \item \textbf{Synthetic Data:} While coordinates are based on the real Dhahran topology, specific demand values and cost parameters were synthetically generated based on industry estimates rather than actual Aramco operational data.
    \item \textbf{Static Model:} The current formulation assumes static demand; future work could incorporate time-varying demand patterns (day/night shifts, seasonal variations).
    \item \textbf{Single-Period Planning:} The model optimizes a single planning period; multi-period capacity expansion models could capture facility construction sequencing more accurately.
\end{itemize}

\subsection{Future Research Directions}
\begin{enumerate}
    \item \textbf{Stochastic Demand:} Incorporate uncertainty in security demand using robust or chance-constrained optimization.
    \item \textbf{Dynamic Response Modeling:} Add explicit travel time modeling with routing constraints rather than simple SLA distance limits.
    \item \textbf{Multi-Objective Optimization:} Balance cost minimization against response time minimization and coverage maximization using Pareto-optimal frontiers.
    \item \textbf{Real-World Validation:} Collaborate with Saudi Aramco to validate the model using actual operational data and conduct pilot deployments.
    \item \textbf{Integration with IoT:} Extend the framework to incorporate real-time sensor data for adaptive resource reallocation.
\end{enumerate}

\subsection{Final Remarks}
The optimization framework developed in this study demonstrates that significant operational cost savings are achievable through strategic facility location and human-robot resource allocation. As autonomous security technologies continue to mature, organizations like Saudi Aramco can leverage such decision-support tools to systematically plan their transition toward Industry 4.0 security operations while maintaining the human oversight essential for ethical and effective decision-making. Source code available at: \url{https://github.com/naimackerman/aramco_security_opt}


% ---------- Highlighting example ----------
\subsubsection{Highlighting Text or Citations}
\begin{highlight}[0.5in]
Text of this category must be italicized, justified, and have .5-inch margins all around. This ensures that important material is highlighted facilitating meaning conveyance.
\end{highlight}

\subsubsection{Recommendations}
We strongly encourage you to use this document as a template for developing your own manuscript.

% ---------- References (actual reference list) ----------
% ---- Format references as shown below. Citations and references must comply with the APA reference style. For the initial paper submission, do not include title or author information in references to previous work by the paper’s author. Include full reference information for the final paper submission.

% ---- Note 1: Begin the Refence section on a new page
% ---- Note 2: If you have a reference manager, use APA 7th.
% ---- Note 3: For the initial paper submission, do not include title or author information in references to previous work by the paper’s author. Include full reference information for the final paper submission

\newpage
\nocite{*}
\section{References}
\printbibliography[heading=none]

% ---------- Biography Format ----------
\newpage
\phantomsection
\makeatletter
\renewcommand{\authorbioentry}[3]{%
  \noindent\begin{tabular}{@{}m{0.5in} M{\dimexpr\columnwidth-0.5in\relax}@{}}
    \authorpic{#1} &
    {\headingfont\bfseries\raggedright\fontsize{12pt}{14pt}\selectfont #2}\par #3
  \end{tabular}\par\medskip
}
\makeatother
% ---------- Biography ----------
% ---- Note : Begin the Biography section on a new page

\section*{Biography}
\authorbioentry{template-images/author1_pic.jpg}{Author Name}{Provide a short biography of the author. Provide a short biography of the author.}
\authorbioentry{template-images/author2_pic.jpg}{Second Author}{Provide a short biography of the second author.}
\authorbioentry{template-images/author3_pic.jpg}{Third Author}{Provide a short biography of the third author.}

\end{multicols*}

\end{document}